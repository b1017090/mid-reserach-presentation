%
% 公立はこだて未来大学卒業研究中間報告書[全コース対応版]
%
%         ファイル名:"sample.tex"
%
%\documentclass[11pt]{jarticle}
\documentclass[11pt]{ujarticle} %%uplatexを用いる際はこちらを使用
\usepackage{funinfosys}
\usepackage{url}
\usepackage[dvipdfmx]{graphicx}
\author{%
b1017090 熊谷峻\\指導教員 : 松原克弥
}
\course{Information Systems Course}
%\course{Advanced ICT Course} %% 高度ICTコースの場合はこちらを使用
%\course{Information Design Course} %% 情報デザインコースの場合はこちらを使用
%\course{Complex Systems Course} %% 複雑系科学コースの場合はこちらを使用
%\course{Intelligent Systems Course} %% 知能システムコースの場合はこちらを使用
\title{VMMによる軽量かつセキュアなアドホッククラウド基盤}
\etitle{Lightweight and secure ad hoc cloud platform with VMM}
\eauthor{Shun Kumagai}
\abstract{近年,BYODやリモートワークの普及により,組織内に余剰な計算資源が散在するようになっている. そこで,この余剰計算資源を活用したクラウドコンピューティング技術である,アドホッククラウドが注目されている. これを取り入れることで,組織内の計算資源の利用率を向上させることができる. しかし,実行中のコードがPCに影響を与えたり,コード実行中のPCがクラウドから離脱することでクラウドジョブの持続が不可能になるといった課題が存在する. この課題に対処するために,本研究では,軽量なVMMであるBitVisorを用いてコードを実行するPCの環境を保護し,FaaS型クラウドを構築することでクラウドからの離脱を許容できるようなアドホッククラウド基盤を提案する.}
\keywords{VMM, アドホッククラウド, FaaS}
\eabstract{In recent years, due to the spread of BYOD and remote working, surplus computing resources are scattered throughout organizations. Therefore, ad hoc cloud computing technology, which is a cloud computing technology that makes use of these surplus computing resources, has been attracting attention. Ad-hoc cloud computing is one of the most popular cloud computing technologies. However, there are some problems, such as the impact of the running code on the PC and the persistence of the cloud job cannot be sustained due to the departure of the PC from the cloud. To deal with these problems, we propose an ad-hoc cloud infrastructure that protects the environment of the PC executing the code using BitVisor, a lightweight VMM, and allows the PC to leave the cloud by constructing a FaaS-type cloud. }
\ekeywords{VMM, ad hoc cloud, FaaS}
\begin{document}
\maketitle
%\vspace*{-.5cm}

\section{情報システムコースにおける本研究の位置づけ}
本研究では、安心・安全・快適な情報社会を支援する観点から、価値ある情報システムの創造、効率性と信頼性を考慮した情報システムの実現、多様で大規模な情報の生成と分析に関する具体的な課題に取り組み、その結果の評価を通じて、新しい方法論や、学問領域を切り拓く能力を育む。

中間報告書中のいずれかの場所に,学生所属コースのカリキュラム・ポリシーに基づき,本研究の位置づけを述べる.

\section{はじめに}
\subsection{背景}
近年,Bring Your Own Device (BYOD) の普及により,会社内に個人のPCを持ち込むことが増えている. そのため,業務を個人のPCで行うことは少なくない. また,リモートワークの増加により,会社に行かずに家で作業することが多くなっている. これら二つの要因により,組織内に利用されていないPCが増加している. これは言い換えると,組織内に余剰計算資源が散財していると言える. この組織の余剰な計算資源を有効活用するため,次世代のクラウドコンピューティング環境としてアドホッククラウドが研究されている. アドホッククラウドは,組織内で十分に活用されていない計算資源を利用して柔軟な計算インフラを作成することができる. また,余剰な計算資源を活用することで,組織内の計算資源の利用率を向上させることができるという利点がある.

\subsection{課題と目的}
アドホッククラウドを構築することで,組織内の余剰資源を活用することができるという利点がある一方で,3つの課題が存在する. まず1つ目に,コードを実行するPCのOS環境に影響を及ぼすことがある. 例えば,実行中のコードによってPCの主プロセスを圧迫したり,クラッシュしてしまったりする可能性がある. そのため,PC環境に影響を与えないようにセキュアにコードを実行しなければならない. 2つ目の課題として,異種OSプラットフォームへの対応がある. これはアドホッククラウドに参加する,組織内にある各PCのOSが異なる可能性があるためである. そのため,異種OSで実現可能なアーキテクチャが必要である. 3つ目の課題として,コード実行中のPCがクラウドから離脱してしまうという課題がある. これは,利用する計算資源がクラウド以外のタスクが増加することにより動的に変化することで起こる. そのため,実行中のPCがクラウドから離脱しても,他に実行中のコードやPCに影響を与えないクラウド環境が必要である. そこで,本研究では,Virtual Machine Monitor(VMM) を用いたFaaS型クラウドを構築することで, 異種OSで実現可能な軽量かつセキュアなアドホッククラウドの仕組みの実現を目指す.

\section{既存技術}
\subsection{コンテナ}
1つ目の既存技術にコンテナがある.コンテナは,Linuxカーネル機能のcgroupやNamespaceを利用することで,CPUやメモリなどのリソースを隔離し,独立したOS環境を作り出すことができる技術である.Docker SwarmやKubernetesといったコンテナオーケストレーションツールを使って操作できる.これによって独立したコード実行環境の実現ができるため,セキュアなコード実行環境の実現ができる.しかし,問題点として,コンテナの実現にLinuxカーネルの機能を使用しているため,Linux系OS以外では基本的に利用不可能である.仮想化を利用して,WindowsやmacOS上でLinux系OSを動作させることにより利用できるが,仮想化によるオーバーヘッドが増加してしまう.FaaS環境を構築できるOSSに,FissionやKubeless,OpenFaaS,OpenWhiskといったものがあるが,どのフレームワークもコンテナ技術をベースとしているため,異種OSに対応することができないという問題がある.

\subsection{プロセスマイグレーション}
2つ目の既存技術として,プロセスマイグレーションがある.プロセスマイグレーションとは別の環境にシステムやデータなどを切り替えることである.コードを実行中のインスタンスをマイグレーションすることにより,クラウドジョブを持続することができるため,コード実行中のPCがクラウドから離脱することを許容できるようになる.しかし,問題点として,アドホッククラウドは動的に変化する計算資源を利用するため,頻繁にマイグレーションが発生してしまい,オーバーヘッドが発生することがある.また,異種OSで実現する場合は,複雑な実装と高度なマイグレーション方法が必要となってしまう.

\subsection{提案}
本研究では,Function as a Service(FaaS) 型クラウドを構築し,軽量なVMMであるBitVisorを用いたセキュアなアドホッククラウド基盤を提案する.

\subsection{Virtual Machine Monitor(VMM)}
本研究では,VMMを利用する.VMMはハイパーバイザとも呼ばれる,ハードウェアとOSの間で動作するソフトウェアである.ホストOS上で動作するホスト型と違って,ハイパーバイザ型は,OSよりも下の層に位置するため,ハードウェアへアクセスする際にホストOSを経由しない.そのため,仮想化によるオーバーヘッドを抑えることができ,OSに依存しないので異種OSで実現可能となる. しかし,OSとハードウェアの通信にVMMが介在することによってオーバーヘッドが発生してしまう.そこで,軽量なVMMであるBitVisorを活用する.
BitVisorは,特定のデバイスのみ仮想化し,必要最低限のI/Oのみを監視・制御する準パススルー型VMMである.また,本来ハイパーバイザは複数のゲストOSで動作するが,その機能が省略されているため,軽量なことが特徴である.このBitVisorを利用することで,仮想化によるオーバーヘッドを軽減する.さらに,BitVisorには,計算資源は共有しつつも,ユーザ環境から隔離された実行環境を実現できる保護ドメインという機能がある.この保護ドメインにインスタンスを作成することで,OS環境へ影響を与えることを防ぐ.

\subsection{FaaS}



\section{先行研究}

中間報告書の文量は4ページとする.学籍番号をファイル名としたPDFファイル1つにまとめた形で作成すること.提出するファイル名はb10xxxxx.pdfとする.

句読点は「,」,「.」とする.「、」,「。」は使用しない.アブストラクトなど英文表記の部分については,スペルチェックプログラムによるチェックをする.

\section{提案する理論}

\subsection{数式}

数式による記述が必要な場合は,式番号を適切に参照しながらまとめること.

\subsection{図・写真}

読者の理解を助けるため,図や表を効果的に利用すること.図のキャプションは

\begin{center}図1 ○○○○\end{center}

のように,図の下に記す.表のキャプションは

\begin{center}表1 ○○○○\end{center}

のように,表の上に記す.

\section{実験と評価}

\section{考察}

\section{結言}

\begin{thebibliography}{99}
\bibitem{marumaru}
	○○△△, システム情報科学会論文誌, 2, 13-19, 2002.
\bibitem{abc}
	A.B.Cdddddd, J. Systems Information Science, 11, 1145-1159, 2001.
\bibitem{batubatu}
	○○××, □□△△, システム情報科学, ☆☆出版, 1999, 20-21.
\bibitem{efghij}
	E.Fggg and H.Ijjj, Electrical Engineering, KKPress, 2003, 281-284.
\end{thebibliography}
\end{document}
%
%
% EOF
