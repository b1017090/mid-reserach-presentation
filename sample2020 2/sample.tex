%
% 公立はこだて未来大学卒業研究中間報告書[全コース対応版]
%
%         ファイル名:"sample.tex"
%
%\documentclass[11pt]{jarticle}
\documentclass[11pt]{ujarticle} %%uplatexを用いる際はこちらを使用
\usepackage{funinfosys}
\usepackage{url}
\usepackage[dvipdfmx]{graphicx}
\author{%
b1017090 熊谷峻\\指導教員 : 松原克弥
}
\course{Information Systems Course}
%\course{Advanced ICT Course} %% 高度ICTコースの場合はこちらを使用
%\course{Information Design Course} %% 情報デザインコースの場合はこちらを使用
%\course{Complex Systems Course} %% 複雑系科学コースの場合はこちらを使用
%\course{Intelligent Systems Course} %% 知能システムコースの場合はこちらを使用
\title{VMMによる軽量かつセキュアなアドホッククラウド基盤}
\etitle{Lightweight and secure ad hoc cloud platform with VMM}
\eauthor{Shun Kumagai}
\abstract{近年,BYODやリモートワークの普及により,組織内に余剰な計算資源が散在するようになっている. そこで,この余剰計算資源を活用したクラウドコンピューティング技術である,アドホッククラウドが注目されている. これを取り入れることで,組織内の計算資源の利用率を向上させることができる. しかし,実行中のコードがPCに影響を与えたり,コード実行中のPCがクラウドから離脱することでクラウドジョブの持続が不可能になるといった課題が存在する. この課題に対処するために,本研究では,軽量なVMMであるBitVisorを用いてコードを実行するPCの環境を保護し,FaaS型クラウドを構築することでクラウドからの離脱を許容できるようなアドホッククラウド基盤を提案する. }
\keywords{VMM, アドホッククラウド, FaaS}
\eabstract{In recent years, due to the spread of BYOD and remote working, surplus computing resources are scattered throughout organizations. Therefore, ad hoc cloud computing technology, which is a cloud computing technology that makes use of these surplus computing resources, has been attracting attention. Ad-hoc cloud computing is one of the most popular cloud computing technologies. However, there are some problems, such as the impact of the running code on the PC and the persistence of the cloud job cannot be sustained due to the departure of the PC from the cloud. To deal with these problems, we propose an ad-hoc cloud infrastructure that protects the environment of the PC executing the code using BitVisor, a lightweight VMM, and allows the PC to leave the cloud by constructing a FaaS-type cloud. }
\ekeywords{VMM, ad hoc cloud, FaaS}
\begin{document}
\maketitle
%\vspace*{-.5cm}

\section{背景と目的}

このサンプルは情報システムコースにおける中間報告書の様式について説明したものである.必ずしもこの雛形を使う必要はないが,仕上がりイメージはできる限りこの雛形にあわせること.

用紙サイズはA4,向きは縦とし,上下の余白は30mm、左右の余白は25mmとする.本文には明朝体とTimes New Romanを用いる.ただし,タイトルや章節の見出し,図表のキャプションはゴシック体とする.タイトルは14ポイント,氏名と章の見出しは12ポイント,節の見出しは11ポイント,その他は10ポイントとする.また,和文タイトルから英文キーワードまでは1段,本文は2段で構成とし,1段のセクションは42文字×45行,2段のセクションは20文字×45行とする.

なお,章立てはあくまでも参考であり,これに限らない.

\section{◯◯コースにおける本研究の位置づけ}
中間報告書中のいずれかの場所に,学生所属コースのカリキュラム・ポリシーに基づき,本研究の位置づけを述べる.

未来大学のカリキュラム・ポリシー
\url{https://www.fun.ac.jp/curriculum-policy} のうち,学生所属コースの項に書かれている卒業研究に関する記述を参照.

\section{関連研究}

中間報告書の文量は4ページとする.学籍番号をファイル名としたPDFファイル1つにまとめた形で作成すること.提出するファイル名はb10xxxxx.pdfとする.

句読点は「,」,「.」とする.「、」,「。」は使用しない.アブストラクトなど英文表記の部分については,スペルチェックプログラムによるチェックをする.

\section{提案する理論}

\subsection{数式}

数式による記述が必要な場合は,式番号を適切に参照しながらまとめること.

\subsection{図・写真}

読者の理解を助けるため,図や表を効果的に利用すること.図のキャプションは

\begin{center}図1 ○○○○\end{center}

のように,図の下に記す.表のキャプションは

\begin{center}表1 ○○○○\end{center}

のように,表の上に記す.

\section{実験と評価}

\section{考察}

\section{結言}

\begin{thebibliography}{99}
\bibitem{marumaru}
	○○△△, システム情報科学会論文誌, 2, 13-19, 2002.
\bibitem{abc}
	A.B.Cdddddd, J. Systems Information Science, 11, 1145-1159, 2001.
\bibitem{batubatu}
	○○××, □□△△, システム情報科学, ☆☆出版, 1999, 20-21.
\bibitem{efghij}
	E.Fggg and H.Ijjj, Electrical Engineering, KKPress, 2003, 281-284.
\end{thebibliography}
\end{document}
%
%
% EOF
